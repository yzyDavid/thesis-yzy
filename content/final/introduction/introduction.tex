\cleardoublestylepage{common}

\section{绪论}

\subsection{背景}

本设计的标题是大规模在线服务应用的资源调度解决方案,这个题目有些宽泛,但是我们可以找到合理的切入点,在一个小角度上作出自己的创新与探索。

这个题目的提出来源于工业界真实面对的问题,在近年来的互联网应用和云服务应用场景下,应用程序的部署规模和复杂程度日益提升。在这个背景下,如何完成资源分配,调度和服务治理,使得资源利用率提高,节能,降低成本,同时保持足够优良的服务可用性和其他指标,就成为了一个日益重要的主题。

而这个前提下,互联网的规模化效益与云计算的应用场景,催生了应用部署的平台化。在本论文中,平台化被定义为,运行不同数据或业务逻辑的一类相似或相同的应用程序。这个定义下文中会详细解释。

\subsubsection{实际问题}

工业界

平台化的应用程序有如下几个案例:

具有不同租户和大量数据库与表的数据库系统,为了节约成本,优化性能,不同的租户会分享相同的数据库服务进程。

云服务中的弹性计算服务,或者更广义的云服务器,虚拟机。如果将虚拟机程序看作英语,那么不同租户运行的副本都是不同的业务与数据,而应用程序相同。

深度学习推理集群。

\subsubsection{调度系统与容器编排}

\subsubsection{业界已有的解决方案}

\subsubsection{业界仍在探索的问题}
