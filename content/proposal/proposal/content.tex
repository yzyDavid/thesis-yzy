\section{项目的主要内容和技术路线}

\subsection{主要研究内容}

本论文的研究内容是提出一些资源调度,管控,服务治理上面的经验与技术总结,探索,并且给出一些算法。同时给出一个工程实现,以解决这方面的部分问题,并且可以直接应用。

\subsection{技术路线}

在容器编排,资源调度等领域中,已经有不少的开源或者私有的实现。已知这些框架的存在,并且在很多足够大而且复杂的场景下经受住了考验,说明此领域是充分可行的。本文的工程实现也将选择站在这些巨人的肩膀上,尽可能以某种调度框架的扩展的方式来完成本文的目标。下面列出我们参考或者依赖的实现。

Kubernetes

Kubernetes 是由云原生基金会所维护的项目,其设计思想脱胎于Google内部使用的Borg集群管理系统。Borg系统介绍的发表于OSDI的论文翻译已经附在本文末尾。Kubernetes借助于容器的实现,负责容器编排与资源调度分配。

Kubernetes系统主要由kubelet,API server与kube-proxy等部分组成。他们常以不同的权限和方式进行部署。

Kubelet是集群系统的agent进程。需要裸运行在集群中的机器上,并且属于集群的每一台服务器上都要运行。他负责听从“大脑”的指示,接收并执行集群的管理命令。

API Server 是最终用户向k8s集群提交并管理任务使用的RESTful API接口。K8s的API对象使用方式非常接近声明式风格,即你所提交的不是命令而是你想得到的最终状态。API server负责暴露这个操作接口,真实数据会被另作存储,然后由系统中的管理调度模块进行实际的操作。

Cotroller manager 与 scheduler 是k8s集群真正执行任务的管理者。即是上文提到的“大脑”。

Kube-proxy 是管理网络代理的部分。Kubernetes建议容器之间使用虚拟网络进行通信。集群管理好容器的虚拟IP地址等细节,然后每个部分之间通信即可尽可能不考虑网络层面的事情,继而不考虑服务发现,熔断限流等等问题,将这些问题统一管理。本论文所考虑的问题中,这部分不是重点。

Kubeflow
kubeflow是基于kubernetes框架二次开发的一个机器学习领域的集群管理工具套件。它使用了kubernetes operator等技术。封装了包括但不限于tensorflow等深度学习框架的训练,测试与推理服务等部分的功能。简而言之,使用它你可以将大部分深度学习相关的工作在kubernetes集群上完成。

Apache YARN
YARN是一个资源调度系统,它主要在Apache Hadoop生态圈当中被广泛使用。

\subsection{可行性分析}

上一个部分之中,我们列举出几种与本文有关的工程实现,由这些成熟产品的实现来证明本文的实现的可行性。本部分我们从可行性这个方面再做一些论证,然后讲解一下kubenetes operator,分析一下使用operator技术来完成此项目的工程开发部分的可行性。

Kubernetes Operator是一种将更多调度和操作的权限暴露给用户的规范。正如名字所表示的那样,Operator所模拟的就是一个用户定义的运维工程师。他的实现依赖于Kubernetes暴露的两个概念。自定义资源定义(CRD)与自定义控制器。

CRD

CRD(自定义资源定义)指k8s暴露给用户添加自定义的资源类型在API对象中的功能。如果不提供这个功能,用户只能将一切想要得到的目标抽象为k8s本身提供的几种资源和部署类型。这样用户只能将一切调度操作交给原生的controller,并且难以满足一切想要的抽象,更不可能完成应用正常运行所需要的业务逻辑。可以说这个时候的k8s自带的调度设施主要只能为无状态的应用服务,典型例子是web后端服务或者类似而广义一些的微服务。他们的典型特征是实例除了可能读取配置,不会向本地存储有其他的读写需求,主要通过读写数据库,消息队列,RPC与其他服务交互等形式工作并提供服务。所以每个实例往往是等价的,可以随时启停的。
对于部分有状态的服务,k8s提供了一些基本的资源,比如编号和远程持久化存储等等。他们对于zookeeper,etcd这样的典型分布式协调服务也许还能工作,对于更重些更复杂的应用,就几乎不能处理了。

对于基本的controller不能处理的情况,CRD允许了用户用自定义的资源类型进行描述,并且可以方便的被操作,被认知。这是承载其他复杂应用的基础。

自定义控制器

自定义控制器即是在kubernetes的概念中,用户自行提供的处理pod中容器生命周期等事务的组件。这允许了用户改变kubernetes对pod进行调度的基本行为。这个接口的暴露直接将kubernetes生态的灵活程度提高了一个维度,使k8s从执行调度与容器编排的角色转变为提供平台的角色。在这两者的基础上,Operator也就呼之欲出了。

Operator

Operator所做的事情即是读取API对象,这自然可以包含用户自定义的资源对象。然后对这些用户指定的目标负责,将其驱动到用户所期望的状态工作。这其中的逻辑可以说全部由Operator负责。所以给了使用者很大的灵活性。

这样说来,我们不一定能理解为何需要Operator,怎样的应用不能被k8s系统自己简单直接地解决呢?这里举一个例子。假设你维护了一个分布式数据库集群,需要自动化管理集群的扩缩容,灾难恢复等等事项。此时数据库的一个节点(即一台机器上的服务进程)宕机不可用了。你要做的事情包括,更改整个集群的配置,启动新进程替换已故障的节点,触发数据恢复,重新平衡等等事宜。这一系列操作k8s平台本身是不可能知晓的,而我们可以把这些人工处理步骤编写为operator自动运行。K8s集群为我们分配硬件资源,其上层的运维操作由我们编写的代码来完成。

对于本设计而言,我们可以使用Operator这种技术来完成两件事情,首先是应用系统本身的正常运行,其次我们可以通过合理设计API对象以及接口,将集群信息和业务目标分开设计,其中的调度编排关系交由Operator来处理。
