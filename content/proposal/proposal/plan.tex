\section{研究计划进度安排及预期目标}

\subsection{进度安排}

本次毕业设计的构想从2018年就开始了,现将时间安排整理列出如下:

\subsubsection*{现状调研}

2018.11 - 2019.2

本阶段查阅公开技术文档和资料,完成了对资源调度,容器编排等服务系统设计与实现的调研。

\subsubsection*{方案设计与构想}

2019.2 - 2019.3

本阶段进行本项目的设计。

\subsubsection*{可行性分析与开题报告撰写}

2019.3 - 2019.4

本阶段将已经有初步构想的设计方案具体推理清楚,给出一个可行性分析,并对预演这个开发过程和工作流程进行思考,同时将思考的一些过程和结果写入开题报告中。

\subsection*{开发}

2019.4 - 2019.6

到了本阶段,准备工作都已经基本完成,只要专注于将已有的设计转化为实现,验证效果即可。在开发的过程当中,我们还有搭建环境和编码实现两件事情要做。

搭建环境对于本项目来讲,复杂度远高于其他项目。因为无论使用何种方式,模拟出来多台机器的集群环境都是复杂的操作,并且要保证资源隔离,网络连接等等步骤都不能出差错,一旦出现问题,调试方法也难有定法,更是可能拖延工期的危险特征。

开发阶段,本项目可能存在的问题是难以调试。作为资源调度相关的程序,固然可以将调度数据的输入和输出进行抽象,从而对调度的输出预期进行假设来调试,此种抽象之下调度过程是一个纯函数。然而一旦 bug 扩散到这个过程之外,调试的难度就会陡然上升。因为你只能构造环境去复现真实场景下运行的调度结果,去分析完整系统的行为,而非分析一个每一步可以追溯的纯过程。因此在开发编码的过程当中,一定要注意质量控制,避免出现问题。

下面我们列出开发阶段要做的几件事:

\subsubsection{搭建环境}

这一阶段,我们将会搭建开发环境以及 Kubernetes 集群环境,并且测试这一切都可以正常工作。比如可以跑通 demo 等。

\subsubsection{适配框架}

这一阶段,我们先实现一个 demo Kubernetes Operator 调度器,让他可以在框架上工作,跑通,而不实现算法细节。

\subsubsection{实现调度器}

这一阶段,我们实现一个与调度细节剥离开的调度算法,即一个输入调度信息,输出调度决策的纯函数,并对其进行测试。

\subsubsection{实现执行器}

这一阶段,将上面的部分与实际框架调度部分相结合,完成这一模块。

\subsubsection{联调测试}

这一阶段,进行一些实际的测试,使上面开发的内容可以实际工作。

\subsubsection{给出结论}

进行后文所提到的对比测试,给出量化指标以支撑我们的结论。

\subsection{预期目标}

预期目标是将一些已有的,考虑多租户隔离,资源利用率与服务容灾能力的调度模型以合理的方式进行复现,并达到初步可用的能力。

我们希望这个项目提出的方法相对而言简单易用,并且能在一定场景下发挥出真实的价值。

下面从不同的方面详细论述一下我们对这个项目的预期。

\subsubsection{工程实现}

工程实现的目标是将我们提出的算法和调度模型在一个已有的资源调度和编排系统,如我们暂定的 Kubernetes 上设计并开发完成。我们预期这个设计与 Kubernetes 已有的生态可以有机地融合在一起,并且可以简单且顺利地被其他人利用。

\subsubsection{结果对比}

找到一些类似的用例,对比采用我们开发的系统能够提升多少资源利用率,或者提升多少服务进程数以期提高容灾能力。

为了能够量化本项目的实际效果,我们需要选取一些有价值的可以量化的指标,下面我们列举一些指标的类型,在项目结束之后观察。

QPS

Queries per Second,每秒可以服务的请求数。这个指标可以反映整个集群的服务负载能力,或者单个应用单个业务的服务负载能力。我们可以通过压力测试等方式实验出在可以承受的延迟下,我们系统部署的应用的服务负载能力,从而得出我们的系统对于在线服务类应用的服务能力和可用性能做到多大提升。

服务可用性 - 故障恢复时间

对于在线服务,任何时长的服务中断都应当被视为一次故障。那么我们的系统设计可以对故障抗性以及故障发生之后的恢复速度有怎样的改变呢?这也可以通过实验来得知。我们可以搭建一个小型实验集群,随机地关闭其中一些机器或者断开部分网络,从而观察:1,系统能否继续服务,对于不同的业务和租户,是否一致,影响面积有多广;2,若服务中断,需要多久才能恢复,恢复时需要多少人工干预。

\subsubsection{实际使用}

本项目的目的在于实际环境下提升资源利用率,因此最好找一个试点的地方运行一段时间,观察效果和稳定性。
