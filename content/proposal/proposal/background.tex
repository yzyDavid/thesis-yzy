\section{问题提出的背景}

% \par 正文格式与具体要求\autocite{zjuthesisrules}

\subsection{背景介绍}

在近年来的互联网应用和云服务应用场景下,应用程序的部署规模和复杂程度日益提升。在这个背景下,如何完成资源分配,调度和服务治理,使得资源利用率提高,节能,降低成本,同时保持足够优良的服务可用性和其他指标,就成为了一个日益重要的主题。

\subsubsection{项目提出的原因}

在大规模的平台化应用部署场景下,我们往往面对一个这样的问题。不同的租户使用相同或者类似的应用程序,但是使用不同的业务数据或者逻辑。不同的应用个体对各种形式的资源需求量各不相同。在这个前提下进行高效合理的资源调度与分配,同时满足资源利用率高,容灾能力强等条件便成为了一个问题。

\section{本研究的意义和目的}

本研究的意义和目的在于探索这样的应用场景下,资源使用率,容灾能力等等指标更高的资源调度方式,并且给出一个在某些场景下可用的工程实现。