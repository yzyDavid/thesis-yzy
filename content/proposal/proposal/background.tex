\section{问题提出的背景}

% \par 正文格式与具体要求\autocite{zjuthesisrules}

\subsection{背景介绍}

在近年来的互联网应用和云服务应用场景下,应用程序的部署规模和复杂程度日益提升。在这个背景下,如何完成资源分配,调度和服务治理,使得资源利用率提高,节能,降低成本,同时保持足够优良的服务可用性和其他指标,就成为了一个日益重要的主题。

\subsubsection{项目提出的原因}

在大规模的平台化应用部署场景下,我们往往面对一个这样的问题。不同的租户使用相同或者类似的应用程序,但是使用不同的业务数据或者逻辑。不同的应用个体对各种形式的资源需求量各不相同。在这个前提下进行高效合理的资源调度与分配,同时满足资源利用率高,容灾能力强等条件便成为了一个问题。

\section{本研究的意义和目的}

本研究的意义和目的在于探索这样的应用场景下,资源使用率,容灾能力等等指标更高的资源调度方式,并且给出一个在某些场景下可用的工程实现。

目前的计算机工程业界,越来越多的调度场景正在向规模化,平台化等等形态变化。以往的传统人工的,粗放的调度方式将不再容易满足我们的需求。

一方面,随着规模的扩大,人力成本将随之水涨船高,同时我们假定每一次操作和每一台服务器,每一个部署都有一个固定的故障率。这个概率本身可能很小,但是随着人工操作数量的提高和规模的扩大,概率将逐步提高到一个不可忽略的程度。

另一方面,随着规模的扩大,服务器以及周边设施等等成本的提高也十分显著。在这个量级上,任何节约资源的方式做到的一个很小的节约百分比,都可能节约难以估量的成本。

在例如浙江大学这样的大规模学术机构面前,各个学院所需的计算能力总和规模巨大。长期以来,浙江大学缺少一种从全局统筹计算资源使用的方式。资源利用率在不同学部,学院甚至实验室之间都严重不均。而无论是经济预算的问题,还是技术问题,或者其他类型的原因,学校都没有一个从全局视角统筹规划资源使用的方式。现状往往是各个实验室使用自己的预算采购高性能计算机,这些高性能工作站只能被零散地放置在实验室内,缺少一个安全稳定的工作环境。不慎绊倒一下电源线,或者洒一杯水在机器上,就会让他的工作被马上中断甚至受到永久破坏。同时,资源利用的峰值特性不均,实验室有大量的批处理类型的任务,因为大部分的科学实验都是处理计算大量数据的单个任务,计算完毕资源就马上闲置下来。这样导致大部分机器都在满负载和空负载的周期之间轮转,平均利用率总是有瓶颈。

尽管很多学校有学校层面的超级计算机集群等计算设施,这种层面上的资源共享仅仅是做到了不同学院,实验室等部门共用同一个资源池,资源使用的峰值特性,弹性承受负载的能力依然得不到提高。可是毕竟统一计算设施才有更大的发挥可能,比如学校层面可以按实际计算资源使用量向各部门收取租用预算,从而做到更精确的资源用量控制,以后其他的创想才可能实现。

