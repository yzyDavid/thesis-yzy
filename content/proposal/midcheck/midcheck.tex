\cleardoublepage
\chapter{中期检查}

\section{项目概况}

本项目初步完成了开发环境搭建,并且开始了调度器的开发与调优工作。

\section{工作进展情况}

\subsection{Kubernetes 环境搭建}

Kubernetes 集群的基本运行需要以下几个部分:

\subsubsection*{etcd}

etcd 的官方描述称之为 etc distributed。即它的功能目标是将 UNIX 系统目录的 /etc 分布式,中心化起来。实际上它的功能与 Apache Zookeeper 类似,是一个分布式高可用的键值对存储服务,或者称为分布式协调服务。在 Kubernetes 当中,一份 k8s 的系统部署会将集群的所有信息持久化在 etcd 当中。所以一套可用的 k8s 部署一定会有一个 etcd 集群。

\subsubsection*{kube-apiserver}

API Server 是用户与 k8s 集群的桥梁,负责接受用户提交的任务,将其持久化并等待调度。

\subsubsection*{kube-controller-manager}

Controller 是 k8s 集群中负责调度某一个应用部署的角色,他们本身需要被一个 Manager 去管理,这个 Manager 也就间接起到了管理整个集群的功能。同样的这个管理者也需要能获取 k8s 集群的部署信息,因此他也需要连接 API Server。可以见得 API Server 一定是承担了以 kubernetes 领域专用的格式和权限管理约定暴露整个集群所有数据的功能。

\subsubsection*{kubelet}

kubelet 是需要部署在受集群管控的每一台机器上的,它是集群 worker 的 agent 角色。kubelet 以 HTTP Server 的方式工作,接受管理者发出的请求来执行命令。

\subsubsection*{kube-proxy}

\subsection{快速搭建方式}

因为从零开始搭建 kubernetes 集群可能存在很多细节问题,也有很多条件需求,对域名,网络等等条件都有依赖,且工作量和调试难度都不低。因此这里我们也介绍一下一些已有的工具,用来快速搭建集群。出于不同的目的,两种方式我都进行了尝试。

\subsection{Operator 开发}

本部分讲述在开发 Operator 的过程,遇到的问题,设计思路以及其中的细节。

\subsection{算法设计}

\section{问题与建议}

\section{其他}
