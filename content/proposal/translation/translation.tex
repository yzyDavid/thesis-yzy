\cleardoublepage
\chapter{外文翻译}

Large-scale cluster management at Google with Borg

Google 使用 Borg 系统进行的大规模集群管理

\section*{摘要}

Borg是谷歌内部使用的集群管理系统。在这个系统的管理之下,有数千个应用的数十万级别的任务在运行。他们横跨了多个达到上万台服务器规模的集群。

Borg通过进程级别隔离的机器资源共享,资源超卖,任务组合等方式极大地提升了资源利用率。它通过调度策略降低了应用整体故障宕机的风险,并且通过运行时的特性最小化故障恢复时间,从而为高可用应用提供可能。Borg提供了描述任务的DSL(领域专用语言),服务发现机制,实时任务监控及相关的分析工具来简化用户的操作。

本文提供了Borg系统的架构和功能的简述,同时包括了重要的设计决定和策略的分析,以及十年间使用经验留下的经验和教训。

\section{节标题}

简介

我们称之为Borg的系统负责谷歌内部所有应用的提交,调度,启动,重启与监视,下面讲解这一切是如何做到的。

Borg主要有三个优点:

1、隐藏了资源管理和故障恢复的细节,使用户可以专注于业务应用。

2、操作的可靠性和可用性高,从而使应用一样获利。

3、使任务负载可以高效地在上千台服务器的集群规模下部署。

Borg不是第一个完成这些功能的系统,但可能是唯一一个可靠性与完成度足以在如此大的规模下运行的。本文也将从这几个方面入手论述。

用户的视角

Borg的用户是运行Google的应用和服务的开发者与系统管理员。用户以“工作”(job)的形式将任务提交给Borg系统,每个工作包括一个或多个运行相同应用程序(二进制)的“任务”(task)。每个工作将运行在一个“单元”(cell)当中。单元指一组行使相同职能的机器。

工作负载

Borg cell 运行的异质负载分为两个主要部分。首先是长期运行的服务,应该“永远”不被停止,并处理对延迟敏感的请求(几μs到几百ms)。 这些服务用于面向最终用户的产品,例如Gmail,Google Docs和网络搜索和内部基础设施服务(例如BigTable)。 第二个是从少数人那里获得的批量任务几秒到几天完成; 这些任务对短期性能波动不太敏感。任务的比例在不同的cell中各有区别,根据它们的主要租户不同,运行的业务类型区别也很大。批处理作业启启停停,许多面向终端用户的服务表现出了昼夜使用量不同的模式。这些场景Borg都能很好地处理。

Borg架构

Borg单元由一组机器,一个称为Borgmaster的逻辑集中控制器和一个代理程序组成,称为Borglet,它在一个单元格中的每台机器上运行。 Borg的所有组件都是用C ++编写的。

可用性

失败是大规模系统的常态,图3提供了15个样本单元上的故障原因分析。 预计在Borg上运行的应用程序使用多副本等技术处理此类事件,将持久状态存储在分布式文件系统中,并且(如果适当的)偶尔记录检查点。 即便如此,我们也会尝试减轻这些事件的影响。

利用率

Borg的主要目标之一是有效利用谷歌的机器舰队,在如此之大的机器金融投资面前,提高利用率几个百分点可以节省数百万美元。 本节讨论并评估一些Borg采用的策略和技术。

隔离

我们50%的机器运行9个或更多任务; 大约90%的机器有大约25个任务,将运行大约4500个线程。 尽管在应用程序之间共享计算机可以提高利用率,但它还需要良好的机制来防止任务相互干扰。 这同时对于安全性和性能都有价值。

相关工作与引用

资源调度这个话题已经被研究了数十年了,涉及的领域包括高性能计算(网格计算),工作站网络和服务器集群。我们这里列出一些和Borg场景最接近的,也即服务器集群相关的工作和研究。下面列出一些已知的开源软件和内部系统。

Apache Mesos

它的设计分离了资源管理和分配放置功能,资源管理是中心化的而调度分配有框架化的实现。

YARN

Apache YARN是Hadoop生态中的调度系统。

Facebook Tupperware

Tupperware公开的细节不多,比较接近Borg。

Microsoft Autopilot

此系统运行了微软的软件监视,部署和维护功能。

Alibaba Fuxi

伏羲系统支持了数据分析类的负载,其调度思路与Borg有所不同。 
